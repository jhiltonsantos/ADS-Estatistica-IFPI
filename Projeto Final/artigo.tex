\documentclass[14pt, a4paper]{article}

\usepackage[utf8]{inputenc}
\usepackage[T1]{fontenc}
\usepackage{amsmath}
\usepackage{amsfonts}
\usepackage{amssymb}
\usepackage{a4wide}
\usepackage{color}
\usepackage{indentfirst}



\begin{document}
	\begin{center}
		\underline{\textbf{ACIDENTES AÉREOS NO BRASIL}}\linebreak

		\textit{FOCO DE ACIDENTES AÉREOS NAS PRINCIPAIS ROTAS DO PAIS}\linebreak 	
	\end{center}

	\begin{enumerate}
	\item \textbf{INTRODUCAO}
	\newline
	O mercado de voos domésticos no Brasil ultrapassou o número de 103 milhões 		no ano de 2018. No mesmo ano, no dia 25 de Junho, o site de rastreamento de 	transportes aéreos, o FlightRadar24, mostrou que, em um único dia, houve 		230.00 voos operando em todo o mundo. Com o crescimento de voos também veio 	o de acidentes, com um aumento de 36\% em 2018 no estado de São Paulo. \\
		Nosso projeto tem como motivo avaliar o crescimento de acidentes aéreos 	do período de 2008 até 2018 utilizando uma base de dados de ocorrências 		aeronáuticas gerenciada pelo CENIPA (Centro de Investigação e Prevenção de 		Acidentes Aeronáuticos), disponível em:  https://www.kaggle.com/nosbielcs/		opendataaigbrazil.\\
		Com a base de dados poderemos avaliar as principais rotas do país, 			avaliar os principais causadores de acidentes, quais as principais 				aeronaves e o nível de desastre provocado. 
		 	
	
	
	
	\item \textbf{PROBLEMA DA PESQUISA}\\
	\newline
	Nosso projeto tem como foco analisar a quantidade de acidentes no território brasileiro e entender qual a sua relação com os tipos de acidentes e os principais modelos de aeronaves. 
	
 
	
	
	
	\item \textbf{OBJETIVOS}\newline
	\\
	Identificar, através dos anos, quais os principais tipos de acidentes em cada região do país, e observar as rotas com maiores números de ocorrências.
	
	\underline{Para isso iremos levantar:} 
	\begin{itemize}
	\item Apontar qual a relação entre a quantidade de acidentes e as rotas;
	\item Identificar quais os principais tipo de acidentes em cada região do país;
	\item Observar características das aeronaves, como: ano de fabricação, modelo, tipo de motor, quantidade assentos, fabricante. E sua relação com o tipo de acidente.   
	\end{itemize}		
	
	
	
	
	\item \textbf{JUSTIFICATIVA}\newline
		
	Temos proposta avaliar as principais situações de acidentes, assim podemos identificar se as ocorrências tem como principais causadores quais tipos de aeronaves; aviões comerciais; aviões de pequeno porte; helicópteros; e outros tipos, qual a atividade que a aeronave estava exercendo. Favorecendo a avaliação nesses casos.
	
	
	\item \textbf{REVISÃO TEÓRICA}\newline
	A abordagem do tema "foco de acidentes aéreos nas principais rota do país", envolvendo diversos assunto, bem como os estudos de análises estatísticas dos dados tratando do conteúdo proposto. Este capítulo abordará referências científicas envolvendo conceitos desta pesquisa.
	
	
	5.1 Centro de Investigação e Prevenção de Acidentes Aeronáuticos
	
	O CENIPA(Centro de Investigação e Prevenção de Acidentes Aeronáuticos) surgiu em 1971. A criação do CENIPA tem como intuito a realização de  investigações e promover a prevenção de acidentes aéreos, de acordo com as normas internacionais.
	
	5.1.1 Objetivo do CENIPA
	
	Atualmente o CENIPA tem como atribuições a supervisão, a coordenação de investigações dos acidentes, o controle e prevenção dos acidentes aéreos. Além disso desenvolve anualmente atividades operacionais e regulamentares.
	
	5.2 Histórico de acidentes
	No estado de São Paulo entre os anos 2000 e 2005, aconteceram 74 acidentes da aviação geral. Dos quais 38 desses acidentes não haviam sido divulgados os relatório pela CENIPA . Os acidentes estão diretamente ligados a uma das seguintes categorias: condições prévias de atos inseguros (17,6),  influências organizacionais (18,1) , supervisão insegura (28,3) e atos inseguros (36) ¹ "\textit{Valores em percentuais}".
	    
	
	\item \textbf{METODOLOGIA}\newline
	\item \textbf{CRONOGRAMA}\newline
	\item \textbf{ORÇAMENTO}\newline
	\item \textbf{BIBLIOGRAFIA}\newline
	\item \textbf{ANEXOS}\newline
	\item \textbf{REFERÊNCIAS}\newline
	 1. FAJER, M; ALMEIDA, I. M.; FISCHER, F. M. Fatores contribuintes aos acidentes aeronáuticos. Revista de Saúde Pública, SP, Brasil. Disponível em: https://www.scielosp.org/scielo.php?pid=S0034-89102011000200024&script=sci_rttext&tlng=en
	\end{enumerate}

	


\end{document}
